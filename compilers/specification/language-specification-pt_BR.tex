\documentclass[a4paper,11pt]{article}
\usepackage[T1]{fontenc}
\usepackage[utf8]{inputenc}
\usepackage{lmodern}
\usepackage{listings}
\usepackage[brazilian]{babel}

\title{Especificação da Linguagem de Programação SAN}
\author{Daniel San Ferreira da Rocha}

\begin{document}

\clearpage\maketitle
\thispagestyle{empty}

\pagebreak

\tableofcontents

\pagebreak

\section{Introdução}
A linguagem de programação SAN foi desenvolvida para ser uma linguagem procedural, estaticamente
e fortemente tipada. É permitido a criação de funções para uma melhor organização dos códigos
desenvolvidos com esta linguagem.

\section{Características Gerais}
O início de todo programa escrito em SAN é a função principal (entry-point) \texttt{init}.
Outras funções escritas devem estar definidas antes da função principal.

As variáveis declaradas terão um escopo local e toda a instrução deve conter o caractere que
delimita seu termino, que é o \textbf{ponto e vírgula}.

Para delimitar escopo é usado o \texttt{"\{"} como início e o \texttt{"\}"} para finalizar.

Também é permitido na linguagem SAN a inserção de comentários em uma linha, para isso basta
iniciar seu comentário com \texttt{"\#"} e tudo que estiver após será desconsiderado.

\section{Tipos de dados}
Ao declarar uma variável, seu identificador deve obrigatoriamente se iniciado com uma letra
(maiúscula ou minúscula), após isso é aceito qualquer carácter (sem acentuação), números e
underline, porém é proibido o uso de espaços em branco. Por fim o tamanho do identificador
é inimitado.

Para declarar uma variável na linguagem SAN é necessário usar uma palavra reservada que
corresponde ao tipo de dado desejado (que será apresentado logo a baixo), seguido do
identificador da variável.

\subsection{Palavras reservadas}
\begin{itemize}
    \item \texttt{int} - Inteiro
    \item \texttt{float} - Ponto Flutuante
    \item \texttt{char} - Caractere
    \item \texttt{bool} - Boleano
    \item \texttt{str} - Cadeia de Caracteres
    \item \texttt{array} - Arranjos Unidimensionais
\end{itemize}

\clearpage

\subsection{Coerção}
Por ser uma linguagem fortemente tipada não é permitido a coerção implícita de tipos, exceto para
os tipos \texttt{int} e \texttt{float}. Caso um valor \texttt{float} seja atribuído a uma variável
\texttt{int}, apenas a parte inteira será considerada.

\subsection{Operações}
\begin{table}[h!]
    \centering
    \begin{tabular}{|c|l|}
         \hline
         \textbf{Tipo} & \textbf{Operações Suportadas}\\
         \hline
         \texttt{int} & Atribuição, Aritméticos e Relacionais\\
         \hline
         \texttt{float} & Atribuição, Aritméticos e Relacionais\\
         \hline
         \texttt{char} & Atribuição, Relacionais\\
         \hline
         \texttt{bool} & Atribuição, Relacionais e Lógicos\\
         \hline
         \texttt{str} & Atribuição, Relacionais e Concatenação\\
         \hline
         \texttt{array} & Atribuição, Relacionais\\
         \hline
    \end{tabular}
    \caption{As operação serão definidas na seção seguinte}
\end{table}

\section{Operadores}
\subsection{Atribuição}
Na linguagem SAN, \texttt{=} é o operador de atribuição. É necessário salientar que a atribuição
de um valor a uma variável deve ser feita separadamente da sua declaração. Do lado esquerdo do
operador deve ser fornecido o ID da variável e do lado direito o valor a ser atribuído (que
necessariamente deve ser do mesmo tipo). Segue um exemplo de declaração seguido da atribuição:

\begin{lstlisting}
int a; # Declaration
a = 3; # Initialization
\end{lstlisting}

No momento da declaração a variável assumirá um valor padrão, que segue na tabela abaixo:
\\
\begin{center}
    \begin{tabular}{|c|l|}
         \hline
         \textbf{Tipo} & \textbf{Valor Padrão}\\
         \hline
         \texttt{int} & \texttt{0} (zero)\\
         \hline
         \texttt{float} & \texttt{0.0} (zero)\\
         \hline
         \texttt{char} & \texttt{` '} (espaço em branco, em ASCII corresponde a 32)\\
         \hline
         \texttt{bool} & \texttt{false}\\
         \hline
         \texttt{str} & \texttt{``''} (cadeia de caracteres vazia)\\
         \hline
         \texttt{array} & Valor padrão do seu tipo\\
         \hline
    \end{tabular}
\end{center}

\subsection{Aritméticos}
\begin{center}
    \begin{tabular}{|c|c|}
         \hline
         + & Adição\\
         \hline
         - & Subtração\\
         \hline
         $\ast$ & Multiplicação\\
         \hline
         / & Divisão\\
         \hline
         $\wedge$ & Exponencial\\
         \hline
         $\sim$ & Unário Negativo\\
         \hline
    \end{tabular}
\end{center}

\subsection{Relacionais}
\begin{center}
    \begin{tabular}{|c|c|}
         \hline
         == & Igualdade\\
         \hline
         != & Diferença\\
         \hline
         > & Maior que\\
         \hline
         < & Menor que\\
         \hline
         >= & Maior ou igual que\\
         \hline
         <= & Menor ou igual que\\
         \hline
    \end{tabular}
\end{center}

\subsection{Lógicos}
\begin{center}
    \begin{tabular}{|c|c|}
         \hline
         \& & Conjunção\\
         \hline
         | & Disjunção\\
         \hline
         ! & Negação\\
         \hline
    \end{tabular}
\end{center}

\subsection{Concatenação de Caracteres}
A concatenação de caracteres é possível por meio do operador \texttt{\%}.

\subsection{Ordem de Precedência e Associatividade}
A tabela a seguir irá mostrar a ordem de precedência dos operadores de forma decrescente com o
número da linha assim como a sua regra de associatividade (DE = Direita para Esquerda, ED = 
Esquerda para Direita):
\begin{center}
    \begin{tabular}{|c|c|}
         \hline
         \textbf{Operadores} & \textbf{Associatividade}\\
         \hline
         $\wedge$ & DE\\
         \hline
         $\sim$ & DE\\
         \hline
         +, -, $\ast$, / & ED\\
         \hline
         ==, !=, >, >=, <=, < & ED\\
         \hline
         \&, | & ED\\
         \hline
         = & DE\\
         \hline
    \end{tabular}
\end{center}

\hfill

É permitido alterar a ordem de precedência dos operadores fazendo o uso de parenteses.
Exemplo em: $1 + 2 * 3$, teríamos como resultado \texttt{7}, porém caso necessário podemos
alterar a ordem como no exemplo: $(1 + 2) * 3$, que nos da um resultado \texttt{9}. Isso ocorre
pois o que estiver entre os parenteses será computado primeiro.

\section{Instruções}
As estruturas apresentadas logo a baixo também deveram terminar com um terminador, nesse caso
o \texttt{\$} será usado.

\subsection{Estrutura Condicional de Uma e Duas Vias}
Foi definido as seguintes estruturas: \texttt{if}, \texttt{if/else} e \texttt{if/elif/else}.
Para utilizar a instrução \texttt{if} é necessário fornecer uma condição lógica ou uma variável
boleana \textbf{entre parênteses}, para que em caso afirmativo o seu respectivo bloco de código
seja executado. Em caso negativo ou nada é executado ou o bloco correspondente ao \texttt{else}
será executado.

Caso necessário, ainda é possível usar o recurso do \texttt{elif} para ter validações ``aninhadas''
, que seguem a mesma regra do \texttt{if}. Ou seja, caso a condição lógica seja afirmativa, o seu
bloco é executado, caso negativo, ou nada é executado ou o próximo \texttt{elif} é testado ou
o próximo bloco de \texttt{else} é que será executado.

Abaixo é mostrado um exemplo com todas as possibilidades:
\begin{lstlisting}
if (a == b) {
    # if command block
}$

# ...

if (a == b) {
    # if command block
} else {
    # else command block
}$

# ...

if (a == b) {
    # if command block
} elif (a == c) {
    # elif command block
} else {
    # else command block
}$
\end{lstlisting}

\textit{Obs 1:. É considerado um caso afirmativo uma expressão lógica que tenha resultado verdade,
ou qualquer valor diferente de 0.}

\textit{OBs 2:. É possível ter quantos \texttt{elif} ``aninhas'' se deseje.}

\subsection{Estrutura Iterativa com Controle Lógico}
A instrução para esse objetivo é o \texttt{while}. As instruções contidas no seu bloco de código
serão repetidas enquanto a condição lógica que o acompanha (assim como no \texttt{if} também
fornecida \textbf{entre parenteses}) for verdadeira. Segue um exemplo:
\begin{lstlisting}
while (a == b) {
    # while command block
}$
\end{lstlisting}

\subsection{Estrutura Iterativa Controlada por Contador}
Nessa estrutura, os comandos do seu bloco serão executado de 0 à n vezes, dependendo de que
uma condição de termino seja atingida. Essa estrutura receberá como \textbf{entre parenteses}
variáveis iniciais, a condição limite (de termino) e um incremento opcional das variáveis
iniciadas. Essa estrutura tem por instrução a palavra reservada \texttt{for}. Segue um exemplo:
\begin{lstlisting}
for (i = 0; i < 10; i = i + 1) {
    # for command block
}$
\end{lstlisting}

\subsection{Entrada e Saída}
Esta definido uma função para a entrada de dados, que recebe obrigatoriamente como parâmetro
um identificador de uma variável que receberá o valor lido:
\begin{lstlisting}
read(<identificador>);
\end{lstlisting}

Também esta definido uma função para a saída de dados, que recebe obrigatoriamente como
parâmetro o identificador da variável que se deseja escrever na saída:
\begin{lstlisting}
write(<identificador>);
\end{lstlisting}

\section{Funções}
É possível definir funções para melhor estruturar o código escrito em SAN. Para tal é necessário
definir um \textbf{nome} para a função, logo em seguida, \textbf{entre parenteses}, deve ser
fornecido todos os parâmetros que essa função irá aceitar como entrada, cada um com seu respectivo
tipo, após é obrigatório a declaração de um tipo de retorno.

O retorno de uma função é feito meio da palavra reservada \texttt{return}.

Para se declarar uma função existem algumas restrições:
\begin{itemize}
    \item Não é permitido o ``aninhamento'' de funções, ou seja, cada função deve ser declarada
    separadamente uma da outra;
    \item Não é permitido a sobrecarga de funções, ou seja, é necessário que todas as funções
    tenham um \textbf{nome único};
    \item Toda função deve ser declarada antes da função \textit{init}.
\end{itemize}

Exemplo:
\begin{lstlisting}
add(int a, int b) int {
    return a + b;
}$
\end{lstlisting}

A casos em que as funções não precisam retornar nenhum valor, para isso é possível que essa função
seja declarada com o retorno do tipo \texttt{nothing}. Nesse caso não será necessário o uso do
\texttt{return} ao final da função. \textit{(\texttt{nothing}, é um tipo espécial que apenas é
permitido como retorno de uma função, não sendo possível a declaração de uma variável com esse tipo.)}

Exemplo de função com retorno vazio (\texttt{nothing}):
\begin{lstlisting}
log(str m) nothing {
    write(m);
}$
\end{lstlisting}

\section{Exemplos}

\subsection{Hello World}
\begin{lstlisting}
init() nothing {
    write("Hello World"); # Commentary
}$
\end{lstlisting}

\pagebreak

\subsection{Fibonacci}
\begin{lstlisting}
fibonacci(int n) int {
    int f1;
    int f2;
    int fi;

    if (n <= 0) {
        return 0;
    } elfi (n == 1 | n == 2) {
        return 1;
    }$

    f2 = 1;
    for (1; n - 1; 1) {
        fi = f1 + f2;
        f1 = f2;
        f2 = fi;
    }$

    return fi;
}$

init() nothing {
    int n;
    read(n);
    
    int fib;
    fib = fibonacci(n);
    write(fib);
}$
\end{lstlisting}

\pagebreak

\subsection{Shell Sort}
\begin{lstlisting}
init() nothing {
    write("Hello World"); # Commentary
}$
\end{lstlisting}

\end{document}
